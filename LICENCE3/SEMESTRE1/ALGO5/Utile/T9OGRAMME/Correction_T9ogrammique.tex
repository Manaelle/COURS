\documentclass[10pt]{article}
\usepackage[utf8]{inputenc}
\usepackage{fancyheadings}
\usepackage[francais]{babel}
\usepackage{verbatim}



\begin{document}
\section*{Correction T9ogrammique}
\noindent
\textbf{Description :}  Il s'agit de traduire et corriger un mot, écrit avec un clavier T9 (voir "T9ogrammes").

~\\
Le programme prend en entrée un ensemble de mots (dictionnaire), et une séquence de touches tapées au clavier.
Le programme doit alors déterminer le ou les mots correct(s) le(s) plus proche(s) de la séquence de touches (c'est-à-dire ayant la distance minimale avec la séquence de touches).
~\\
~\\
On définit comme suit la distance entre deux codes T9 :\\
La distance d'un code à un autre est donnée par le nombre minimal de modifications à faire pour passer de l'un à l'autre : ces modifications peuvent être une insertion, une suppression, ou un changement d'une touche.

~\\
\noindent
Sur un clavier T9 "classique" :
\begin{center}
\begin{tabular}{|c|c|c|}
\hline
 1 & 2 & 3\\
 & abc & def\\
\hline
 4 & 5 & 6\\
 ghi & jkl & mno\\
\hline
 7 & 8 & 9\\
 pqrs & tuv & wxyz\\
\hline
\end{tabular}
\end{center}
\noindent
\begin{itemize}
\item "dans" a pour code 3267
\item "sans" a pour code 7267
\end{itemize}
Donc "dans" et "sans" sont à une distance 1 l'un de l'autre (un changement de touche).

~\\
~\\
\noindent
\textbf{Exemple 1 :}
\begin{itemize}
\item Clavier T9 classique
\item Dictionnaire : dans avec sans
\item Séquence de touches : 7268
\item Correction(s) optimale(s) : sans
\end{itemize}

~\\
\noindent
Les codes des mots sont :
\begin{itemize}
\item "dans" : 3267
\item "avec" : 2832
\item "sans" : 7267
\end{itemize}
Donc les distances avec 7268 sont de 2 pour "dans", 4 pour "avec", et 1 pour "sans".
La correction optimale est donc "sans".

~\\
~\\
\noindent
Attention, les mots n'ont pas forcément la même longueur !
~\\
\noindent
\textbf{Exemple 2 :}
\begin{itemize}
\item Clavier T9 classique
\item Dictionnaire : papier boite
\item Séquence de touches : 72437
\item Correction(s) optimale(s) : papier
\end{itemize}

~\\
\noindent
Les codes des mots sont :
\begin{itemize}
\item "papier" : 727437
\item "boite" : 26483
\end{itemize}
Donc les distances avec 72437 sont de 4 pour "boite", et 1 pour "papier" (1 insertion).
La correction optimale est donc "papier".

~\\
Si il existe plusieurs corrections optimales, le programme devra les afficher toutes par ordre lexicographique.

~\\
\textbf{Données lues sur l'entrée standard :}\\
Sur une première ligne, un ensemble de mots constituant le dictionnaire, séparés par des espaces.\\
Sur la deuxième ligne, la séquence de touches, séparées par des espaces.\\
Puis, sur les lignes suivantes, la configuration du clavier ; sur chaque ligne :
\begin{itemize}
\item Le numéro de la touche
\item le caractère ':'
\item l'ensemble de lettres associées à la touche (uniquement minuscules de a à z, non accentuées)
\end{itemize}
~\\
\textbf{Affichage à produire :}\\
Sur une ligne, séparées par des espaces, la (les) correction(s) optimale(s) pour la séquence de touches (triées par ordre lexicographique).
~\\
~\\
\noindent
\textbf{Exemple 1 :}\\
\noindent
Entrée~:
\begin{verbatim}
dans avec sans
7 2 6 8
2:abc
3:def
4:ghi
5:jkl
6:mno
7:pqrs
8:tuv
9:wxyz
\end{verbatim}

\noindent
Sortie~:
\begin{verbatim}
sans
\end{verbatim}
~\\
~\\
\noindent
\textbf{Exemple 2 :}\\
\noindent
Entrée~:
\begin{verbatim}
papier boite
7 2 4 3 7
2:abc
3:def
4:ghi
5:jkl
6:mno
7:pqrs
8:tuv
9:wxyz
\end{verbatim}

\noindent
Sortie~:
\begin{verbatim}
papier
\end{verbatim}
~\\
~\\
\noindent
\textbf{Exemple 3 :}\\
\noindent
Entrée~:
\begin{verbatim}
ping pong long
2 6 6 4
2:abc
3:def
4:ghi
5:jkl
6:mno
7:pqrs
8:tuv
9:wxyz
\end{verbatim}

\noindent
Sortie~:
\begin{verbatim}
long pong
\end{verbatim}
\end{document}
