\documentclass[10pt]{article}
\usepackage[utf8]{inputenc}
\usepackage{fancyheadings}
\usepackage[francais]{babel}
\usepackage{verbatim}



\begin{document}
\section*{T9ogrammes}
\noindent
\textbf{Description :}  Il s'agit de déterminer si deux mots sont des T9ogrammes.\\
Deux mots sont des T9ogrammes s'ils s'écrivent avec les mêmes touches sur un clavier T9.\\
Par exemple, pour un clavier T9 "classique" :
\begin{center}
\begin{tabular}{|c|c|c|}
\hline
 1 & 2 & 3\\
 & abc & def\\
\hline
 4 & 5 & 6\\
 ghi & jkl & mno\\
\hline
 7 & 8 & 9\\
 pqrs & tuv & wxyz\\
\hline
\end{tabular}
\end{center}

\noindent
Le mot "caribou" s'écrit en appuyant sur 2(c) 2(a) 7(r) 4(i) 2(b) 6(o) 8(u).\\
Le mot "abricot" s'écrit en appuyant sur 2(a) 2(b) 7(r) 4(i) 2(c) 6(o) 8(t).\\
Donc "caribou" et "abricot" sont des T9ogrammes.

~\\
\textbf{Données lues sur l'entrée standard :}\\
Sur une première ligne, les deux mots à tester, séparés par un espace.\\
Puis, sur les lignes suivantes, la configuration du clavier ; sur chaque ligne :
\begin{itemize}
\item Le numéro de la touche
\item le caractère ':'
\item l'ensemble de lettres associées à la touche (uniquement minuscules de a à z, non accentuées)
\end{itemize}
~\\
\textbf{Affichage à produire :}\\
Sur une ligne,
\begin{itemize}
\item Si les deux mots forment un T9ogramme : "oui" suivi du code de chaque lettre séparés par des espaces
\item Si les deux mots ne forment pas un T9ogramme : "non"
\end{itemize}
~\\
\noindent
\textbf{Exemple 1 :}\\
\noindent
Entrée~:
\begin{verbatim}
caribou abricot
2:abc
3:def
4:ghi
5:jkl
6:mno
7:pqrs
8:tuv
9:wxyz
\end{verbatim}

\noindent
Sortie~:
\begin{verbatim}
oui 2 2 7 4 2 6 8
\end{verbatim}
~\\
\noindent
\textbf{Exemple 2 :}\\
\noindent
Entrée~:
\begin{verbatim}
coucou salut
4:ac
42:olt
421:us
\end{verbatim}

\noindent
Sortie~:
\begin{verbatim}
non
\end{verbatim}
\end{document}
